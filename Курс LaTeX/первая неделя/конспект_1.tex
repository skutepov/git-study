
\documentclass[a4paper,12pt]{article} % добавить leqno в [] для нумерации слева

%%% Работа с русским языком
\usepackage{cmap}					% поиск в PDF
%%\usepackage{mathtext} 				% русские буквы в формулах
\usepackage[T2A]{fontenc}			% кодировка
\usepackage[utf8]{inputenc}			% кодировка исходного текста
\usepackage[english,russian]{babel}	% локализация и переносы

%%% Дополнительная работа с математикой
\usepackage{amsmath,amsfonts,amssymb,amsthm,mathtools} % AMS
\usepackage{icomma} % "Умная" запятая: $0,2$ --- число, $0, 2$ --- перечисление

%% Номера формул
%\mathtoolsset{showonlyrefs=true} % Показывать номера только у тех формул, на которые есть \eqref{} в тексте.

%% Шрифты
\usepackage{euscript}	 % Шрифт Евклид
\usepackage{mathrsfs} % Красивый матшрифт

%% Свои команды
\DeclareMathOperator{\sgn}{\mathop{sgn}}


%% Перенос знаков в формулах (по Львовскому)
\newcommand*{\hm}[1]{#1\nobreak\discretionary{}
{\hbox{$\mathsurround=0pt #1$}}{}}

%%% Заголовок
\author{Автор}
\title{Конспект первой недели курса \LaTeX{} в Вышке}
\date{\today}

\begin{document} % конец преамбулы, начало документа
	\maketitle
\section{Программные средства}
\begin{itemize}
	\item 	TeXStudio, TeXShop, WinEdt. - редакторы 
	\item TeXLive - система компиляции 
	\item WriteLaTeX и ShareLaTeX. - облачные редакторы
	\item Конвертер в doc или rtf формат LaTeX2rtf, например, или pandoc.  рекомендовали GrindEq math
		
\end{itemize}
\section{Формат документа}
 слово, которое идет  после обратного слэша \textbackslash , нужно воспринимать как  команду к действию
 
 Каждый файл, который вы будете делать, начинается с команды \textit{documentclass}.
 
 
 \textbackslash documentclass[a4paper,12pt]\{article\} 
 
 У этой команды  есть два аргумента:
 \begin{itemize}
 	\item во-первых, говорите, что это за документ, в моем случае, это документ
 	класса article - статья.
 	\item А, во-вторых, говорите сразу некоторые
 	атрибуты текста, которые вы хотите, чтобы в этом документе
 	были.
 	\begin{itemize}
 		\item Во-первых, вы хотите, чтобы документ был
 		на бумаге формата А4
 		\item Во-вторых, вы хотите, чтобы в этом документе,
 		основной кегль шрифта был 12 пунктов.
 	\end{itemize}
 \end{itemize}
\subsection{Классы документов}
\begin{itemize}
	\item article -- обычная статья  -- то простейший формат, в котором, во-первых, поля на всех страницах	одинаковые.
	\item book -- книга -- главы, колонтитулы, симметричные поля
	\item beamer -- слайды презентации
	\item report -- отчет -- главы, колонтитулы
\end{itemize}
\subsection{Структура документа}
LaTeX очень хорош во всем, что касается работы со
структурой документа. И он позволяет, во многом, этот процесс
автоматизировать. То есть, вы пишите специальные команды, а
дальше система сама заботится о том, как нужно оформить то,
что вы написали.
\begin{itemize}
	\item part - Самая большая, самая главная
	единица структурная в документах LaTeX. - это как бы тома книги
	\item chapter (только в книгах (book) и отчетах (report))
	\item section - раздел 	или параграф
	\item subsection - подраздел
	\item subsubsection - подподраздел
	\item paragraph - абзац
	\item subparagraph - подобзац
	
\end{itemize}
\subsection{Тело документа}
\% - комментарии 

Текст документа начинается с
команды  \textbackslash begin\{document\}, и заканчивается командой
 \textbackslash end\{document\}.
 
 Первая команда, которая идет здесь, - это
 команда \textbackslash maketitle. Команда \textbackslash maketitle выведет в
 PDF-документ те исходные данные про автора, заглавие и дату,
 которые вы ввели выше.
 
\subsection{Метки}
 о есть, если что-то в документе нумеруется, например, главы, параграфы,
 уравнения, рисунки, таблицы, то вы можете поставить на него ссылку, так, что эта ссылка будет сформирована
 автоматически. Что я имею ввиду: скажем, есть у вас
 какое-нибудь уравнение, рядом с этим уравнением, или там может быть
 название главы, или рисунка, таблицы. Рядом с этим уравнением вы можете
 написать специальную команду \textbackslash label, у команды \textbackslash label один аргумент в
 фигурных скобках. Это имя этой команды. То есть вы можете, там,
 например, если вы хотите ссылаться на разделы, то
 удобно писать \textbackslash label	
 
 Если меток много, то лучше указывать тип: \textbackslash label\{sec:survey\},
 \textbackslash label\{th:pifagor\}, \textbackslash label\{eq:gdp\}, \textbackslash label\{fig:tree\}
 \begin{itemize}
 	\item th - теорема
 	\item eq - математическое выражение
 	\item fig - рисунок
 	\item вообще любой значения, поясняющее на какой объект ссылка (таблица, диаграмма, скриншот)
 \end{itemize}
 
 \textbackslash usepackage\{showkeys\}  -  вы можете прямо
 рядом со всеми этими теоремами, разделами,
 таблицами, на которые вы ссылаетесь, прямо рядом с ними
 увидеть, что у той или иной таблицы стоит метка. Иногда, это помогает уже при окончательной
 правке документа проверить, что все ссылки стоят
 корректно.
 
\begin{itemize}
	\item На определенную где-то метку можно сослаться
	\begin{itemize}
		\item \textbackslash ref\{имя\_метки\} -- номер объекта;
		\item \textbackslash pageref\{имя\_метки\} -- номер страницы;
		\item \textbackslash eqref\{имя\_метки\} -- ссылка на уравнение.
	\end{itemize}
	\item Пакет varioref позволяет подставлять «на следующей странице» и т. п.
	\item Пакет hyperref позволяет делать гиперссылки.
\end{itemize}
\section{Служебные файлы}
Если вы будете делать презентацию, или картинки вставлять, или рисовать картинки
сами, или делать оглавление, список литературы,
то файлов этих здесь может быть больше
десятка.

файл \textit log ~ это файл, в котором записано все-все-все, что
происходило во время компиляции. Это написано на специальном языке, если вы когда-нибудь станете продвинутым
пользователем, то, может быть, вам будет интересно изучить эти
файлы, особенно если происходят какие-то
непонятные вам ошибки.

Файл \textit aux, auxiliary, создан \LaTeX для того, чтобы быть картой документа, то
есть если в вашем документе есть какие-то перекрестные ссылки или нужно
помнить номера страниц, на которых находятся какие-то объекты, то файл \textit aux содержит эти сведения. 

 Файл \textit synctex позволяет, - позволяет
 в каком-нибудь месте получившегося PDF-документа, вот в этой
 программе просмотрщика, нажать правую кнопку и перейти в место, которое ему
 соответствует в исходном документе. 

 меню этой программы TeXstudio в
 разделе "Инструменты" есть такой пункт "Очистить
 вспомогательные файлы"
 
\section{Математика}

\subsection{Преамбула}
 пакет \textit{mathtext}, который отвечает за то, чтобы русские буквы в формулах
 отображались. \textit mathtext, поскольку он
 неродной для \LaTeX, то есть изначально не предполагалось, что
 кириллические буквы могут использоваться в формулах, иногда он вызывает проблемы в работе, то
 есть подключение некоторых других пакетов может стать невозможным, если вы используете пакет
 \textit mathtext.
 
 подключен целый ряд пакетов,
 многие из которых начинаются с ams. \textit AMS - это американское математическое
 общество, организация, которая внесла очень большой вклад в
 развитие \LaTeX и написала вот эти пакеты, которые сейчас,
 в общем, стали стандартными и одними из
 самых часто используемых.
 
 пакет \textit icomma, это intelligent comma, умная
 запятая. 
 
 Следующие два пакета ~ это шрифт Евклид,
 euscript и еще один математический шрифт на случай если  латинские и греческие буквы заканчиваются, то
 хочется использовать их же, но написанных другими шрифтами всякими,
 там, курсивными или готическими.
\subsection{Формулы}
Чтобы написать формулу, нужно вставить ее
между знаками доллара. Я могу написать: \$2 + 2 = 4\$. - мы перешли в
математический режим и пишем формулу, а не текст. $2+2=4$
Для выключной формулы использовать \textbackslash [ \textbackslash ] вместо \$
\[2+2=4\]
\subsection{Перенос длинных формул в русской традиции}
	Начало формулы оказалось в первой строке
	этого абзаца, а окончание - в следующей строке этого
	абзаца. Если бы мы писали книжку по-английски, то
	то, что произошло, было бы правильным. То есть просто формула разорвалась,
	оставив знак арифметического действия в предыдущей строке и начав новую строку
	со следующего символа. Однако в русскоязычной традиции набора
	математических формул принято переносить знак арифметического
	действия на новую строку тоже.
	Существует способ, предложенный в пособии Львовского, которое является нашей
	основной книжкой. Способ этот заключается в том, чтобы
	задать вот такую команду. В преамбуле это называется "Перенос знаков
	в формулах по Львовскому". Я не буду вдаваться в подробности, что
	именно здесь происходит. А-а. И если вы еще не стали продвинутым
	пользователем, не советую вам разбираться. Просто научитесь работать с этим. Смотрите. Когда я узнал, что у меня знак плюс после
	единички оказывается в конце строки и разрывает эту строку, то я могу поместить
	этот плюс внутрь аргумента команды \textbackslash hm.
	
	Текст текст текст текст текст текст текст текст текст текст текст $1\hm{+}2+3+4+5+6=21$
	
\subsection{Формула - как "содержательный" объект}
Давайте сейчас сделаем так, чтобы эта формула была содержательным объектом в
тексте, то есть, чтобы она получила номер и чтобы
на нее можно было ссылаться. Для того чтобы это сделать, существует окружение, по-английски -
environment, - если вы будете читать какую-нибудь книжку, то это будет называться environment, -
\textbackslash begin\{equation\}.
\begin{equation}\label{eq:mrmc}
MR=MC
\end{equation}

\eqref{eq:mrmc}  на стр. \pageref{eq:mrmc} --- условие максимизации прибыли.

\subsection{Математические знаки и действия}
Чтобы сделать простую дробь, существует
команда \textbackslash frac, от слова fraction.

 Если вы не хотите, чтобы размер
 цифр уменьшался, то есть, если вы хотите, чтобы четверка и двойка оставались
 такого же размера, как и остальные цифры ну, это бывает нужно, если у вас
 там не четверка и двойка, а какие-нибудь тоже сложные выражения, которые важно,
 чтобы читатель увидел в первозданном виде. То вместо команды frac, именно той,
 которая внутренняя команда frac, можно
 использовать команду dfrac.
 
 много текста с \textbackslash frac $\frac{a+b+c}{d+c*r}+1$ вер дляываш фылдтвмш фы вмл фыаьфыв афывлщ мдофы м щфывщмщыофв
 ыдлварлдфырвадлрфыд софыв оыо фыолдв мдо фдв
 текст \textbackslash dfrac $\dfrac{a+b+c}{d+c*r}+1$ текст фдыолвиадиыил ыфо мфыл 
 фыодви дофыивод ыфд вдфцотдф дофы вод 
 
 \textbackslash frac
 \[\frac{1+\frac{4}{2}}{6}=0.5\]
 
 \textbackslash dfrac
 \[\frac{1+\dfrac{4}{2}}{6}=0.5\]
\subsection{Символы}

в умножении вместо * правильно писать команду \textbackslash times $\times$
\begin{equation}
	\begin{aligned}
		 5 &* 5=25 \\
		 5 &\times 5=25
	\end{aligned}
\end{equation}

 Для команды "не равно", существует
 специальная команда ne которая расшифровывается как non equal
\[2\times2\ne5\]

Скажем, я хочу сказать, что какое-нибудь там множество А пересекает
множество B. Символ пересечения, как вы может быть знаете, это такая "подкова", направленная
рогами вниз. Она напоминает головной убор. И здесь, чтобы ее создать, нужно написать
cap, A  \textbackslash cap B - это значит пересечение множества А
и B.  А объединение с B, объединение - это подкова, направленная вверх, и она
напоминает некую чашу. Здесь она задается командой cup.
\[ A \cap B, A \cup B \]

\subsection{Скобки}
\[(2+3)\times5=25\]
\[(2+\frac{9}{3})\times5=25\]
Вот эта скобка, которая рядом с 9/3,
закрывающаяся скобка, она явно не подходит к тем цифрам, рядом с
которыми она стоит.

LaTeX умеет автоматически подбирать высоту
скобок в соответствии с тем, рядом с чем, они
оказались. Для этого нужно сделать вот такую
несложную вещь. Там, где у меня была открывающаяся скобка,
ведь ее размер тоже нужно подобрать. Открывающаяся скобка, я напишу команду
left, прямо перед этой скобкой. То есть была просто скобка, а стала left
скобка. И в закрывающейся скобке напишем команду
right.
\[\left(2+\frac{9}{3}\right)\times5=25\]

фигурные скобки, это очень
важный объект в LaTeX и по умолчанию он воспринимается как
аргумент какой-нибудь команды.  Поэтому, если я хочу показать фигурные
скобки, то мне нужно поставить бэкслэш перед каждой
из них.

\subsection{Стандартные функции}
 LaTeX знает многие стандартные функции,
 которые вам могут пригодиться. Если вы откроете пособие Львовского, то
 там есть глава, посвященная набору математики, в ней есть раздел, в котором
 перечислены функции, поддерживаемые LaTeX. Скажем, если я хочу написать,
 что sin x = 5, Мне достаточно написать команду sin. $\sin(x)=5$
 То же самое можно сделать с косинусом,
 тангенсом, котангенсом, многими функциями, которые
 вы привыкли использовать в математике. Скажем, если мне нужен натуральный
 логарифм, я могу написать здесь просо ln,
\subsubsection{Свои команды}
 если вам нужна какая-нибудь функция,
 которая не известна LaTeX, то есть это какая-то специфическая
 функция, которую используют в вашей отрасли науки или вы сами ее
 придумали для своей научной статьи. Тогда можно сделать вот как, у меня в преамбуле здесь есть пример того, как это
 сделано. "Свои команды" это называется. Есть такая команда DeclareMathOperator,
 
 Свои команды
 \textbackslash DeclareMathOperator\{\textbackslash sgn\}{\textbackslash mathop\{sgn\}\}
\subsection{Диакритические знаки}
А что, если мне нужно чтобы не просто икс, а икс с чертой было равно пяти, с чертой
сверху. Тогда я могу использовать команду bar
просто перед командой икс, я напишу \textbackslash bar x = 5.
\[\bar x = 5\]
Если икс с чертой равно пяти, а икс
с тильдой или с волной сверху равен восьми, то нужно так и
написать - \textbackslash tilde икс равна восьми. Запустим, и диакритический знак уже другой -
это тильда.
\[\tilde x=5\]
Если я хочу, чтобы черта стояла сразу над
несколькими символами, тогда фигурные скобки мне понадобятся
обязательно.
\[\bar{x+r+v+b+a}=5\]
\[\tilde{x+r+v+b+a}=5\]
Если я хочу черту подлиннее, то вместо
команды bar, нужно использовать другую команду,
overline, линия сверху. То же самое с тильдой. Если мне нужна длинная тильда, над
каким-нибудь длинным выражением, то обычный символ,
который мы использовали до этого, приведет к тому же нехорошему результату, который был и с
командой bar. Но есть специальная команда, которая,
которую очень просто запомнить widetilde - широкая
тильда.
\[\overline{x+r+v+b+a}=5\]
\[\widetilde{x+r+v+b+a}=5\]

\subsection{Буквы других алфавитов}
 Давайте попробуем написать что тангенс
 альфа равен единице. И я пишу команду тангенс. Теперь мне нужна буква альфа. Буква альфа пишется очень просто. Есть команда- alpha. Вот и все. Она пишется так же, как и названи этой
 греческой буквы.
\[\tg \alpha = 1 \]
Если вам нужна заглавная какая-нибудь
буква, скажем, заглавная буква фи, то нужно просто написать её с
заглавной буквы. То есть команда, команда, которая задает
строчную, команда, которая задает строчную фи - это три маленькие буквы
phi, которые делают букву фи. А команда, которая задает заглавную фи, -
это те же самые три буквы, только первая из них
тоже заглавная.
\[\tg \Phi = 1 \]
если вам нужна заглавная буква
альфа, давайте попробуем. Что, если я хочу получить заглавную букву
альфа, то есть я напишу альфа с большой буквы. Запустим, и я поздравляю вас, впервые мы
получили ошибку. Ошибка показана в нижней части экрана
в программе TeXstudio. В других программах она может быть в
других местах. Здесь написано: undefined control
sequence \$\textbackslash Alpha. Это значит, что LaTeX неизвестна такая
команда, как алфа с большой буквы. Казалось бы, довольно простая команда. Почему она неизвестна LaTeX? А потому, что вам никогда не понадобится
заглавная греческая буква альфа. Поэтому в LaTeX нет специальной команды
для неё. 
\[\tg A = 1 \]
Оставшиеся два нюанса связаны с тем, как
две греческие буквы принято писать в научных статьях
на разных языках. Вот есть такая буква как эпсилон, которую
вы возможно знаете. Если я напишу, да, давайте я сразу обе
напишу: эпсилон, а вторая - эта та самая фи, только
строчная.
\[\epsilon, \phi\]
Возможно, вы привыкли видеть эти буквы
другими, если вы привыкли читать русские книжки по
математике. Для того чтобы написать буквы эпсилон и фи
в привычном для русского читателя виде, нужно использовать
команды varepsilon, то есть вариация эпсилон, и varphi. 
\[\varepsilon, \varphi\]
\subsection{формулы в несколько строк}
Типичным разделителем строк в LaTeX является просто пустой абзац, как мы с
вами уже изучили. Если вы оставляете пустую строку, то то, что
вы будете писать после этого, LaTeX начнет с
новой строки. Однако в математическом режиме это не
работает.
\[
2\times2=4
3\times3=9
\]
\[2\times2=4\]
\[3\times3=9\]
\subsection{Очень длинные формулы}
очень такая длинная формула, в которой есть сумма натурального
ряда чисел с первого номера по сотый. Кстати, это равно, кажется, 5050. Если я хочу сообщить своему читателю вот
такие сведения о том, что сумма всех чисел от одного до ста
равна 5050, - причем я хочу сообщить это читателю
именно так, как здесь написано, то есть хочу написать начало этого ряда,
потом поставить многоточие, которое ставится командой dots, потом середину
этого ряда, а потом конец.
\[
1+2+3+4+5+6+7+\dots+50+51+52+53+54+56+57+\dots+96+97+98+99+100=5050
\]
 Что получается? Получается не очень хорошо. Во-первых, вся формула как бы сжалась. Вы видите, что расстояние между плюсами
 отсутствует. Это пришлось сделать LaTeX, потому что
 если поставить нормальные пробелы между плюсами и цифрами, то формула бы просто не
 влезла в страницу. 
 Чтобы это сделать, нужно воспользоваться
 окружением. То есть начиная с команды begin и
 заканчивая командой end. begin, окружение multline. Соответственно, в конце нужно написать
end\{multline\}. multline. Что это окружение сделает? Правильный способ отображения очень
длинных формул, то есть тот, который принят в наборе книг, вот какой: вы должны первую
строку прижать к левой части страницы, все строки, кроме первой и последней,
выровнять по центру, а последнюю строку прижать к
правой части страницы.
\begin{multline}
1+2+3+4+5+6+7+\dots+ \\ +50+51+52+53+54+56+57+\dots +\\ +96+97+98+99+100=5050
\end{multline}
\subsection{Несколько формул}
Формулы в несколько строк приходится
оформлять не только в этом случае, а в том случае, если вы хотите поместить
несколько связанных друг с другом и логически, логически
связанных. Тех, которые должны находиться рядом друг с другом,
формул. Вот в какой-то блок между блоками текста. Для этого существует целый ряд команд. Я предлагаю начать с команды, которая
называется align.  Внутри содержимого я могу написать свои
уравнения, которые мне нужны. ут новая новая строка делается точно так
же, как и в multline, то есть двойным
бэкслэшем.
\begin{align}
2\times 2=4\\
3\times 3=9\\
10\times 65646 = 656460
\end{align}
На что можно обратить внимание? Во-первых, каждая формула получила номер. То есть каждая строчка получила номер. В следующем фрагменте мы обсудим, как это
изменить. То есть как более тонко работать с
номерами. Но если вам нужно, чтобы каждая строчка
получила номер, то ничего менять не нужно. Во-вторых, они выровнены по правому краю. Особенность окружения align заключается в
том, что оно позволяет добиваться правильного
выравнивания, собственно, align и переводится, как выравнивание. Выравнивание, как правило, в LaTeX, всякие
табуляторы и вообще, объекты, находящиеся в вертикальном соответствии друг с другом,
ставятся с помощью команды амперсанд. Хорошим тоном является выравнивать,
вот, группу формул, подобной этой, по знаку "равно", то есть чтобы знаки
"равно" оказывались один под другим. Я ставлю амперсанд (\&) в том месте
каждой строчки, где я хочу, чтобы было вертикальное соответствие, перед
знаком "равно".
\begin{align}
2\times 2&=4\\
3\times 3&=9\\
10\times 65646 &= 656460
\end{align}
Окружение align позволяет сделать
еще более мощную вещь, а именно, сделать, чтобы не только
было несколько формул в разных строчках, но и
сделать целую таблицу из формул, то есть сделать
несколько столбиков. Чтобы перейти к следующему столбику
формул, мне нужно еще раз поставить амперсанд. Обратите внимание, все нечетные
амперсанды отвечают за - то есть первый, третий, пятый - отвечают за
выравнивание внутри столбцов. Все четные амперсанды - второй, вот который
я сейчас поставил, дальше четвертый, шестой и так далее - отвечают за
новый столбец формулы.
\begin{align}
2\times 2&=4 & 6\times 8 &=48\\
3\times 3&=9 & A+C&=D\\
10\times 65646 &= 656460 & \frac{3}{2}&=1,5
\end{align}
Здесь у меня не очень красиво получилось,
вот, вы можете видеть, - иногда, когда вы будете писать текст, который вам кажется вполне нормальным, потом смотреть,
что получилось - иногда получаются какие-то нюансы,
которые вам бы хотелось исправить. Вот здесь, когда я смотрю на этот фрагмент,
мне хочется исправить вот эту дробь три
вторых, потому что вы можете видеть, что из-за нее последняя
строка разъехалась, и вот здесь, в первом столбике, разрыв
получился слишком большой. Это выглядит некрасиво. Поэтому в данном случае более удачно
будет написать дробь три вторых не через команду frac, а просто через
обычный слэш, который означает деление. 
\begin{align}
2\times 2&=4 & 6\times 8 &=48\\
3\times 3&=9 & A+C&=D\\
10\times 65646 &= 656460 & 3/2&=1,5
\end{align}

Что, если вам нужно, чтобы не каждая строка
в формуле, которую вы пишите, получала номер, а чтобы
вся группа строк получила номер. Скажем, вы пишите формулировку какой-то
задачи, там есть какая-нибудь целевая функция,
ограничение, и все они относятся к одной задаче,
которой вы хотите этот номер присвоить. Для этого есть окружение с очень похожим синтаксисом, который
называется aligned.
\begin{equation}
\begin{aligned}
2\times 2&=4 & 6\times 8 &=48\\
3\times 3&=9 & A+C&=D\\
10\times 65646 &= 656460 & 3/2&=1,5	
\end{aligned}
\end{equation}
В этом отличие окружения align, которое мы
использовали перед этим, от окружения aligned, которое мы пытаемся
использовать сейчас. Дело в том, что, чтобы использовать aligned, нужно перейти в математический
режим. Если я хочу номер, то я тогда напишу
\begin{equation} и \end{equation} вставлю в конец. То есть, когда я пишу просто align, то LaTeX понимает, что нужно перейти в
математический режим. Когда я пишу aligned, он необязательно это
понимает. Почему так сделано? Потому что aligned, то есть вот эта табличка с
номером, она может быть внутри какой-то более
сложной формулы. То есть кроме этой группы, которой
вы хотите присвоить единый номер, в вашей формуле может быть еще что-то, например,
вы можете поместить окружение aligned внутри
окружения align. И поэтому окружение aligned
воспринимается как часть математического режима, и нужно перейти в специальный математический
режим, чтобы он работал. Сейчас, когда я ввел эти begin\{equation\},
end\{equation\}, давайте запустим. И вы видите, что теперь у меня эти же самые формулы, но получили теперь
единый номер.
\subsection{Система уравнений}
Следующий способ, когда вы
захотите получить формулу, состоящую из нескольких строчек, - это
система уравнений.
Чтобы получить систему уравнений, нам
нужно воспользоваться окружением aligned. Как вы помните, фигурная скобка
задается с помощью специальной команды, то есть вы не можете просто
набрать ее с клавиатуры. Нужно набрать ее с бэкслэшем. Вот я набрал ее перед началом формулы. Вы видите, что TeXstudio, редактор, в
котором я работаю, предложил мне автоматически
закрывающуюся фигурную скобку. Вот она как раз мне не понадобится. У системы только с одной стороны есть
фигурная скобка. Поэтому, я ее удаляю. Я ее удаляю. TeXstudio это не нравится: как вы
видите, он выделяет красным, потому что открывающаяся скобка
есть, а закрывающейся нет. Кроме того, мне нужно, чтобы эта фигурная
скобка была нужной высоты. С этой задачей мы тоже уже сталкивались. Я могу написать здесь left - как вы
помните, когда я пишу left, это создает фигурную скобку или
любую другую скобку, нужной высоты. Ту скобку, которая следует за командой
left. Однако, LaTeX не нравится, что открывающаяся скобка есть, а
закрывающейся нет. Если мы сейчас попробуем это обработать,
то будет ошибка. Это сделано для того, чтобы вы сами
следили за своим набором, то есть, может, вы случайно пропустили скобку,
и тогда вам нужно об этом напомнить. Чтобы избежать случайных пропусков скобки,
LaTeX ругается каждый раз, когда вы его
пропускаете. Нужно четко сказать LaTeX, что мы ничего не пропустили. Ну,
естественным было бы написать здесь right и
закрывающуюся фигурную скобку. И тогда у нас была бы система, которая с двух сторон обрамлена фигурными
скобками: слева и справа. Но мы не хотим этого, нам не нужно, чтобы здесь было две
фигурных скобки. Нам нужна только одна. Поэтому вместо фигурной скобки после
команды right мы поставим точку. Точка - это фантом скобки.
\[
\left\{
\begin{aligned}
2\times x&=4 \\
3\times y&=9 \\
10\times z &= 656460 	
\end{aligned}
\right.
\]
Еще один случай, очень похожий на систему уравнений, это кусочное задание
каких-нибудь функций. То есть когда вы можете сказать, что
какая-нибудь функция равна одним значениям при каком-то условии, а при
другом условии равна другим значениям. Там тоже используется фигурная скобка. И вы можете получить это с помощью того метода, который мы только что
обсудили. Но есть специальный метод, который немного упрощает всю процедуру, этот метод
называется cases.
\[
|x| = \begin{cases}
x, &\text{если } x \ge 0 \\
-x, &\text{если } x < 0
\end{cases}
\]
 Если мы находимся внутри математики, как
 перейти в текстовый режим? Существует специальная команда text. Если
 я напишу "если" внутри команды text, то все будет
 хорошо. Ну, почти хорошо. Итак, я оба "если" заключил внутрь команды
 text, и запускаю это обрабатываться.
 
 \subsection{Матрицы}
 
 Набираю команду begin\{pmatrix\}. pmatrix - p-матрица. Сейчас вы увидите, что
 	это означает. А внутри нее использую такой же синтаксис,
 	как использовался в командах align и aligned, то есть амперсанды для разделения
 	столбиков, и двойные обратные слэши для разделения
 	строчек. 
 \[
 \begin{pmatrix}
 a_{11} & a_{12} & a_{13} \\
 a_{21} & a_{22} & a_{23}
 \end{pmatrix}
 \]
 чтобы  поставить нижний индекс, кстати, нужно поставить такое подчеркивание после
 символа, и дальше в фигурных скобках написать то, что должно
 быть в этом индексе. $a_1, a_123, a_{123_{1_{2}}}$
 
  Для этого, для этих вертикальных линий,
  вместо окружения pmatrix, мне нужно написать
  vmatrix.
 \[
 \begin{vmatrix}
 a_{11} & a_{12} & a_{13} \\
 a_{21} & a_{22} & a_{23}
 \end{vmatrix}
 \]
 
 если мне нужна матрица в
 квадратных скобках, то я, опять же, исправлю здесь
 одну букву, и это будет bmatrix. 
 \[
 \begin{bmatrix}
 a_{11} & a_{12} & a_{13} \\
 a_{21} & a_{22} & a_{23}
 \end{bmatrix}
 \]
 Очень просто, если вам не нужны больше
 никакие навороты от этих матриц, скажем, какие-нибудь линии, разделяющие столбцы
 или строки, то вы можете создавать их вот так просто. Если вам нужны какие-то навороты, скажем,
 вам нужна какая-нибудь окаймленная матрица, и требуется провести где-нибудь
 вертикальную, горизонтальную черту, или поставить многоточие сразу на
 несколько строчек и столбцов, то это потребует несколько
 более продвинутой работы. С этим вы можете познакомиться в любом
 пособии, например, в Львовском. 
 
 \subsection{Нумерация формул}
 
 Вот
 смотрите, если вернуться назад в наш документ, там, где у нас было несколько формул, вот это вот,
 целых, целых шесть формул получили три номера, и
 в общем, это не очень естественно. 
 \begin{align}
 2\times 2&=4\\
 3\times 3&=9\\
 10\times 65646 &= 656460
 \end{align}
 Скорее всего,
 такой набор формул нужен просто, чтобы показать
 какой-то набор выражений. Может быть, я никогда не буду на
 него ссылаться, и номера совсем не нужны. Для этого я могу вместо команды align,
 которую я здесь использовал, - вот видите, я в
 исходном тексте вернулся к этому месту, - могу
 использовать команду align со звездочкой. Это частое явление
 в LaTeX, что если вы вместо какой-то команды
 используете ее версию со звездочкой, то просто
 исчезает нумерация. 
  \begin{align*}
  2\times 2&=4\\
  3\times 3&=9\\
  10\times 65646 &= 656460
  \end{align*}
 
  Но что, если я хочу вместо номера
  два что-нибудь другое? То есть, что, если я хочу
  нумеровать вот именно эту конкретную формулу не цифрой, которая
  ей соответствует, а каким-нибудь своим символом или каким-нибудь словом,
  которое мне нужно, чтобы здесь было. Для этого есть такая команда, tag, и
  внутри команды tag нужно написать, что я здесь
  жду. Скажем эту формулу, я хочу назвать как-нибудь буквой, буквой S, потому что
  это сумма. Я написал здесь tag\{S\} - давайте запустим и
  посмотрим, что произошло. Двоечка, которая здесь была, заменилась на
  букву S.
 \begin{multline}
 1+2+3+4+5+6+7+\dots+ \\ +50+51+52+53+54+56+57+\dots +\\ +96+97+98+99+100=5050\tag{S}\label{eq:sum}
 \end{multline}
 то если я на эту формулу c буквой S
 хочу сослаться? Помните, что есть такая команда label? Я у label напишу название eq, потому что это
 equation. Такой тип, тип объекта, и назову его
 сумма - sum. 
 В уравнении \eqref{eq:sum} на странице \pageref{eq:sum} много слагаемых
 
 Некоторые стандарты требуют, чтобы
 пронумерованы были только те формулы, на которые в тексте есть
 ссылки. Но, иногда, это сложно предугадать, какие
 формулы будут со ссылками, а какие нет. Было бы удобнее пронумеровать все
 формулы, потом расставить ссылки в документе на те из них,
 которые нужны, и сделать так, чтобы отображались только
 нужные номера. То есть нумеровались только те формулы, на которые в тексте есть непосредственная
 ссылка. Поскольку это было бы удобнее, вполне
 естественно, что в \LaTeX это реализовано. Если мы вернемся в нашу преамбулу, там
 есть такая команда, которая пока что была под
 комментарием, - вот эта вот. showonlyrefs = true.
 То есть номера исчезнут у всех формул, на
 которые в тексте не встречаются ссылки. Но у тех, на которые встречаются
 ссылки, номера формул останутся. 
 Что, если ваш редактор вдруг вам сказал,
 что нумерация всех формул в вашем документе должна быть
 слева, а не справа. Вот здесь она, как вы могли видеть, была
 всегда справа. Но, если вам нужно нумеровать слева, то в
 LaTeX это делается очень просто. Нужно в самом начале документа,
 там, где у нас есть команда documentclass, добавить
 еще один аргумент. Там сейчас два аргумента: то, что бумага A4, и
 шрифт 12 пунктов. То, что нужно добавить,  - это опция leqno. Она расшифровывается как "слева,
 уравнение, нумеровать". Если мы это сделаем, запустим сейчас
 документ обрабатываться, то все номера переедут в левую часть, там, где мы
 и хотели. Если убрать эту опцию, то номера окажутся
 в правой части, как и раньше.

\end{document} % конец документа