% Этот шаблон документа разработан в 2014 году
% Данилом Фёдоровых (danil@fedorovykh.ru) 
% для использования в курсе 
% <<Документы и презентации в LaTeX>>, записанном НИУ ВШЭ
% для Coursera.org: http://coursera.org/course/latex .
% Исходная версия шаблона --- 
% https://www.writelatex.com/coursera/latex/1.2

\documentclass[a4paper,12pt,leqno]{article} % добавить leqno в [] для нумерации слева

%%% Работа с русским языком
\usepackage{cmap}					% поиск в PDF
%%\usepackage{mathtext} 				% русские буквы в формулах
\usepackage[T2A]{fontenc}			% кодировка
\usepackage[utf8]{inputenc}			% кодировка исходного текста
\usepackage[english,russian]{babel}	% локализация и переносы

\usepackage{showkeys} 

%%% Дополнительная работа с математикой
\usepackage{amsmath,amsfonts,amssymb,amsthm,mathtools} % AMS
\usepackage{icomma} % "Умная" запятая: $0,2$ --- число, $0, 2$ --- перечисление

%% Номера формул
\mathtoolsset{showonlyrefs=true} % Показывать номера только у тех формул, на которые есть \eqref{} в тексте.

%% Шрифты
\usepackage{euscript}	 % Шрифт Евклид
\usepackage{mathrsfs} % Красивый матшрифт

%% Свои команды
\DeclareMathOperator{\sgn}{\mathop{sgn}}

%% Перенос знаков в формулах (по Львовскому)
\newcommand*{\hm}[1]{#1\nobreak\discretionary{}
{\hbox{$\mathsurround=0pt #1$}}{}}

%%% Заголовок
\author{\LaTeX{} в Вышке}
\title{1.2 Математика в \LaTeX}
\date{\today}

\begin{document} % конец преамбулы, начало документа

\maketitle

первый      абзац 


$2+2=4$


2+2=4

второй абзац

выключная формула: \[ 2+2=4 \] продолжаю текст

2,4
$2,4$ $2, 4$
 sdlvh lsd lgsl glks rwe er werh erh glsd glhdlskg h $1+2+3\hm{+}4+5+6+7$
 
\begin{equation}\label{eq:mrrc}
MR=MC
\end{equation}

еще текст
\eqref{eq:mrrc} на стр. \pageref{eq:mrrc} - условия максимизации прибыли 

\section{Нюансы работы с формулами}
\subsection{Дроби} 
много текста с вер дляываш фылдтвмш фы вмл фыаьфыв афывлщ мдофы м щфывщмщыофв
ыдлварлдфырвадлрфыд софыв оыо фыолдв мдо фдв
текст $\dfrac{a+b+c}{d+c*r}+1$ текст фдыолвиадиыил ыфо мфыл 
фыодви дофыивод ыфд вдфцотдф дофы вод 

\[\frac{1+\dfrac{4}{2}}{6}=0.5\]

\subsection{Скобки}

\[(2+3)*5=25\]
\[(2+3)\times5=25\]
\[(2+\frac{9}{3})\times5=25\]
\[\left(2+\frac{9}{3}\right)\times5=25\]


\subsection{Стандартные функции}
$\sin x =5$
$\ln x =5$
\[\sgn x =1 \]

\subsection{Символы}
\[2\times2\ne5\]
\[ A \cap B, A \cup B \]

\subsection{Диакритические знаки}
\[\bar x = 5\]
\[\tilde x=5\]
\[\overline{x+r+v+b+a}=5\]
\[\widetilde{x+r+v+b+a}=5\]

\subsection{Буквы других алфавитов}
\[\tg \alpha = 1 \]
\[\tg A = 1 \]
\[\tg \Phi = 1 \]
\[\epsilon, \phi\]
\[\varepsilon, \varphi\]

\section{формулы в несколько строк}
\[2\times2=4\]
\[3\times3=9\]
\subsection{Очень длинные формулы}
\[
1+2+3+4+5+6+7+\dots+50+51+52+53+54+56+57+\dots+96+97+98+99+100=5050
\]
\begin{multline}
1+2+3+4+5+6+7+\dots+ \\ +50+51+52+53+54+56+57+\dots +\\ +96+97+98+99+100=5050\tag{S}\label{eq:sum}
\end{multline}

\subsection{Несколько формул}

\begin{align}
2\times 2=4\\
3\times 3=9\\
10\times 65646 = 656460
\end{align}

\begin{align*}
2\times 2&=4\\
3\times 3&=9\\
10\times 65646 &= 656460
\end{align*}

\begin{align*}
2\times 2&=4 & 6\times 8 &=48\\
3\times 3&=9 & A+C&=D\\
10\times 65646 &= 656460 & \frac{3}{2}&=1,5
\end{align*}

\begin{align*}
2\times 2&=4 & 6\times 8 &=48\\
3\times 3&=9 & A+C&=D\\
10\times 65646 &= 656460 & 3/2&=1,5
\end{align*}

\begin{equation}
\begin{aligned}
2\times 2&=4 & 6\times 8 &=48\\
3\times 3&=9 & A+C&=D\\
10\times 65646 &= 656460 & 3/2&=1,5	
\end{aligned}
\end{equation}

\subsection{Система уравнений}

\[
	\left\{
		\begin{aligned}
			2\times x&=4 \\
			3\times y&=9 \\
			10\times z &= 656460 	
		\end{aligned}
	\right.
\]
\[
	|x| = \begin{cases}
	x, &\text{если } x \ge 0 \\
	-x, &\text{если } x < 0
	\end{cases}
\]

\subsection{Матрицы}

\[
	\begin{pmatrix}
	a_{11} & a_{12} & a_{13} \\
	a_{21} & a_{22} & a_{23}
	\end{pmatrix}
\]


\[
\begin{vmatrix}
a_{11} & a_{12} & a_{13} \\
a_{21} & a_{22} & a_{23}
\end{vmatrix}
\]


\[
\begin{bmatrix}
a_{11} & a_{12} & a_{13} \\
a_{21} & a_{22} & a_{23}
\end{bmatrix}
\]

В уравнении \eqref{eq:sum} на странице \pageref{eq:sum} много слагаемых

\end{document} % конец документа